% The Expression
% Syntax exception thrown when initializing expression
As we have seen before, MatrixLib contains its own numerical system and support for different types of matrices.
In addition to this, MatrixLib also allows for the representation of numerical values through expressions.
These expressions contain mathematical operations and values that are supported by the MatrixLib numerical system, so rationals and complex values.
An expression could be as simple as $2+2$.

\subsection*{Operations}
To understand how to properly create an expression, one must know the operations of the expression first.
The \class{Operation} enum currently has five different values:
\inlinecode{ADD}, \inlinecode{SUBTRACT}, \inlinecode{MULTIPLY}, \inlinecode{DIVIDE}, and \inlinecode{POW}.
The operators of these values are $+, -, *, /,$ and \^{} respectively.
These five operators represent the same methods that we saw earlier in the \class{Operable} interface.
Each \class{Operation} value has its own string to represent its operator.
Two quantities can be evaluated through this enum as well.
For example, \inlinecode{Operator.MULTIPLY.getResult(a,b)} where $a$ and $b$ are both quantities.
This would be the same as performing \inlinecode{a.multiply(b)}.

\subsection*{Creating an Expression}
These expressions can be created with the \class{Expression} class and instantiated with the static \inlinecode{Expression\#buildExpression(String)} method.
There is no public constructor for the \class{Expression}, but instead, it is required that this builder method.
% Formatting
When a string, is fed, it must have proper syntax and only valid operators.
In essence, the expression must be formatted in such a way that there are never two adjacent operators,
there may not be more right parentheses than left parentheses, and an operator may never start or end an expression.

% Evaluating an expression
\subsection*{The Interpreter}
An expression alone is to no use, for we do not know its value.
This is where the \class{Interpreter} is a vital piece to the \class{Expression}.
All methods within the \class{Interpreter} class are static and package-visible or private.
So an expression can be only be evaluated as a \class{Quantity} by using the \inlinecode{Expression\#evaluate()} method.

% Interpreter process
When the interpreter runs, it utilizes a recursive process while abiding by the order of operations.
If there are no parentheses, the expression is read and the typical order of operations takes place.
If there are parentheses, the Interpreter will first look for pairs of parentheses, finding the deepest set of parentheses.
The sub-expression found within this set of parentheses is then evaluated, and the original sub-expression and parentheses are replaced by this value.
If parentheses are used to multiply, the interpreter will insert the multiplication operator within the expression's string.
The resulting expression is then evaluated recursively until there are no more sets of parentheses.
Thus, the expression can be directly evaluated just as the sub-expressions had been prior.

\subsection*{Variables}
Currently, MatrixLib has no support for variables.
The current reason is that the addition of algebraic variables could propose a great complexity and the task has not yet been taken on.
In addition to this, there would be a great number of algebraic manipulations that would have to be considered as well.
In the future, this task may be taken on; however, MatrixLib currently only supports direct computations.
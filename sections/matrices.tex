Matrices were the main focus when designing MatrixLib.
However, these matrices could have contained any sort of data structure, which is why the numerical system was created.
The library supports any sort of $m\times n$ matrix where $m$ and $n$ are positive integers.
This starts with the abstract class \class{OperableMatrix}.
The \class{OperableMatrix} represents a two-dimensional array of quantities.
It is able to perform basic operations such as changing values within the matrix, row/column swaps, scaling a row/column, taking row sums, etc.
Since the \class{OperableMatrix} is based upon a two-dimensional array, getting/setting row values is faster than getting/setting.
This is due to the array being in the form of \inlinecode{Quantity[rows][row\_values]}.
This is purely a design choice and could easily be reversed if wanted.

\subsection*{The MatrixBuilder}
% Matrix representation, loading/saving
MatrixLib contains a built-in way to parse matrices from strings.
This creates a convenience when loading or saving matrices from files or other use cases.
Utilizing the \inlinecode{OperableMatrix\#toString()} method, a matrix will be returned as a string.
The \class{MatrixBuilder} can be utilized for parsing matrices from strings.
Here, the \inlinecode{MatrixBuilder\#parseMatrix(String)} builds a matrix from a given input.
Of course, if the input is formatted incorrectly, exceptions will be thrown.
Here is an example of the string formatting,

\[
    \begin{bmatrix}
        1 & 2 \\3&4
    \end{bmatrix}
    \iff
    [\{1,2\};\{3,4\}].
\]

It can be seen that the left and right brackets are almost useless information within the string format of the matrix.
Since it is only two characters, they are kept as a convenience and could be used to separate multiple matrices.
Within the project's test folder, there are multiple methods of saving and loading matrices implemented for testing, including compression handling and saving to a CSV file.

\subsection*{Matrix Operations}
%matrix being iterable
The \class{Matrix} class inherits the \class{OperableMatrix} superclass.
Currently, \class{Matrix} is the only subclass of \class{OperableMatrix}; however, it allows for a more modularized approach.
The \class{Matrix} data structure is designed for adding more operations to the matrix and allowing it to interact with other matrices.
These operations include methods such as row echelon form, reduced row echelon form, matrix multiplication, scaling, and performing entry-wise sums between matrices.

\subsection*{Other Types of Matrices}
There are other, more specialized, types of matrices within MatrixLib as well.
The \class{SquareMatrix} class is a subclass of \class{Matrix}.
Unlike its parent, it must have equal dimensions for its column(s) and row(s).
In addition, this class inherits all of the methods from before.
The \class{SquareMatrix} is able to calculate the determinant of the matrix and give a boolean value of its triangularity - the state of being triangular.
The \inlinecode{SquareMatrix\#det()} method creates a clone of the matrix instance, puts the clone into REF, and then takes the product of all pivots.

The second type of specialized matrix is the \class{IdentityMatrix}.
The \class{IdentityMatrix} is a subclass of \class{SquareMatrix}.
This type of matrix will only have ones across its diagonal and all other values are zero.
This class is \textbf{final}, therefore it cannot have subclasses.
This is because the \class{IdentityMatrix} is intended to be immutable.
The identity matrix can be utilized in computations such as obtaining the characteristic polynomial of a square matrix.
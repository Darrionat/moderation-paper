MatrixLib makes use of a custom number system.
This system is centered around the abstract class, \class{Quantity}.
\class{Quantity} is an immutable data structure that represents a mathematical quantity or set of numbers.
It also implements multiple interfaces, including one named \class{Operable}.
The \class{Operable} interface contains the following abstract methods:
\inlinecode{add(Quantity)}, \inlinecode{subtract(Quantity)}, \inlinecode{multiply(Quantity)},
\inlinecode{divide(Quantity)}, \inlinecode{pow(int)}, \inlinecode{negate()}, \inlinecode{zero()}.
Since \class{Quantity} is an abstract class that is inherited,
it actually does not implement the methods from \class{Operable} - besides \inlinecode{pow(int)} which will be mentioned later.
All of these methods return a \class{Quantity} except \inlinecode{zero()} which returns a \inlinecode{boolean}.
General assumptions based upon the method signatures are most likely accurate, but for further detail, see the JavaDocs.

Currently, the \class{Quantity} superclass is inherited in both the \class{Rational} and \class{Complex} data structures.
This polymorphic approach allows for many benefits such as compatibility, using interface methods instead of class methods, etc.
Creating a numerical data structure serves a few key problems: speed of computation, memory usage, and ease of use.
All three will be addressed in the following.

\subsection*{Rational}
The \class{Rational} is the backbone of the numerical system within MatrixLib.
Rational numbers are the foundation for other numerical systems in MatrixLib.
Within MatrixLib, the \class{Rational} class is designed for speed and ease of use.
Since the \class{Rational} will serve as the foundation for any other numerical systems within MatrixLib, it needed to promise speed.
While using the operational methods, this class directly accesses private class fields instead of ``getters/setters'' in order to access information promptly.
This class is based upon \class{BigInteger}, a utility provided by Java which allows more integers to be represented than primitive types.
One of its constructors, for example, is \inlinecode{Rational(BigInteger numerator, BigInteger denominator)}.

The \class{Rational} class allows for simple operations, such as addition, subtraction, multiplication, and division.
In addition to these operations, it also is able to perform exponentiation.
There are a few more, but they are more specific and can be seen within the library.
As you may have already noticed, not all real numbers can be represented.
This includes real numbers such as $\sqrt{2}$.
The lack of implementation for representation of all real numbers is currently due to the lack of necessity.
However, since MatrixLib is open-sourced, it is quite simple for one to contribute a data structure that represents real numbers.

\subsection*{Complex}
The \class{Complex} class is another data type that represents a \class{Quantity}, although it represents imaginary values as well.
The \class{Complex} value can be represented as $a+bi$ where $a$ and $b$ are both \class{Rational}s.
The beautiful part about the representation of complex numbers is that they can also represent a real value.
In a case where $b=0$, the method, \inlinecode{Complex\#toRational()} can always be used - of course, if the complex value is non-real, an exception will be thrown.
Also, the \class{Complex} structure uses the \inlinecode{pow(int)} method from \class{Quantity} - this will be explained soon.
The \class{Complex} data structure is as simple as creating two different rationals and then constructing a new instance.
This data structure also allows for the same sort of operations as \class{Rational}, but with its own object.

\subsection*{Computations}
Having more than one type of \class{Quantity} arises the issue of compatibility.
Therefore, each data structure's operations take into account the type of \class{Quantity} being processed and how to behave from that.
For example, let $x$ be a \class{Rational} instance and $y$ be a \class{Complex} instance.
When \inlinecode{x.add(y)} is executed, the method will return a \class{Quantity} of \class{Complex} typing.

% quantity having a default pow
As previously mentioned, the only operation that \class{Quantity} overrides from \class{Operable} is \inlinecode{pow(int)}.
This is because MatrixLib contains a custom algorithm that handles exponentiation if a data type does already define one (as \class{Rational} does).
Each \class{Quantity} has a \inlinecode{HashMap<Integer, Quantity>} that represents its powers.
When the \inlinecode{pow(int)} method is called, it puts the powers of $0$ and $1$ into the map.
This is because their values are already known, one and the quantity itself, respectively.
It then converts the given \inlinecode{int} into a binary representation of that integer.
For example, $55 = 32+16+(0)8+4+2+1 \implies 110111$.
There are then two loops that run.
The first loops through the binary string, starting from the second to the last character, and moves towards the beginning of the string.
For each character, it will calculate that power of two, regardless of the bit value at that location.
The second then loops through the entire binary starting at the beginning.
The result initially is equal to $1$.
Each iteration then multiplies this result by the calculated value of that position if the bit value is $1$.
After this loop, the result is finished calculating, and returned.

This may seem like a pretty big ``info dump,'' so let's work through an example.
Say we want to calculate $5^{55}$.
As we know from before, $55$ as a binary string is $110111$.
Therefore, $5^{55} = 5^{32} \cdot 5^{16} \cdot 5^4 \cdot 5^2 \cdot 5^1$.
Since we already know that $5^1$ is $5$, we can then just square each power to obtain the next.
As one can see, this process allows for an extremely quick calculation.

Rationals, however, do not use the default method for exponentiation.
% default BigInteger.pow(int)
They instead utilize a method from \class{BigInteger}.
Moreover, this algorithm works especially well with \class{Complex} values.
% this may seem like a lot of nonsense so give example of 2^55
Let $a+bi$ and $c+di$ be two different complex values where $a,b,c,d$ are rationals.
Here it is shown that regardless of the values, we already know the product of the two complex values.

\[
    (a+bi)(c+di) = ac+adi+bci+bdi^2
\]

\[
    \implies ac+bdi^2+adi+bci
\]

\[
    \implies (ac - bd) + (ad + bc)i
\]

Since $a,b,c,$ and $d$ are all rationals, we can quickly compute the product of two complex values.
Thus, a minimal amount of computations are performed, and when they are performed, the computations are efficient.
This is exactly why the exponentiation algorithm in \class{Quantity} can be performed efficiently for complex numbers.
